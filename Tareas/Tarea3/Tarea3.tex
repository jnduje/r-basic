% Options for packages loaded elsewhere
\PassOptionsToPackage{unicode}{hyperref}
\PassOptionsToPackage{hyphens}{url}
%
\documentclass[
]{article}
\title{Tarea 3}
\author{Nicolas Duje}
\date{18/3/2022}

\usepackage{amsmath,amssymb}
\usepackage{lmodern}
\usepackage{iftex}
\ifPDFTeX
  \usepackage[T1]{fontenc}
  \usepackage[utf8]{inputenc}
  \usepackage{textcomp} % provide euro and other symbols
\else % if luatex or xetex
  \usepackage{unicode-math}
  \defaultfontfeatures{Scale=MatchLowercase}
  \defaultfontfeatures[\rmfamily]{Ligatures=TeX,Scale=1}
\fi
% Use upquote if available, for straight quotes in verbatim environments
\IfFileExists{upquote.sty}{\usepackage{upquote}}{}
\IfFileExists{microtype.sty}{% use microtype if available
  \usepackage[]{microtype}
  \UseMicrotypeSet[protrusion]{basicmath} % disable protrusion for tt fonts
}{}
\makeatletter
\@ifundefined{KOMAClassName}{% if non-KOMA class
  \IfFileExists{parskip.sty}{%
    \usepackage{parskip}
  }{% else
    \setlength{\parindent}{0pt}
    \setlength{\parskip}{6pt plus 2pt minus 1pt}}
}{% if KOMA class
  \KOMAoptions{parskip=half}}
\makeatother
\usepackage{xcolor}
\IfFileExists{xurl.sty}{\usepackage{xurl}}{} % add URL line breaks if available
\IfFileExists{bookmark.sty}{\usepackage{bookmark}}{\usepackage{hyperref}}
\hypersetup{
  pdftitle={Tarea 3},
  pdfauthor={Nicolas Duje},
  hidelinks,
  pdfcreator={LaTeX via pandoc}}
\urlstyle{same} % disable monospaced font for URLs
\usepackage[margin=1in]{geometry}
\usepackage{color}
\usepackage{fancyvrb}
\newcommand{\VerbBar}{|}
\newcommand{\VERB}{\Verb[commandchars=\\\{\}]}
\DefineVerbatimEnvironment{Highlighting}{Verbatim}{commandchars=\\\{\}}
% Add ',fontsize=\small' for more characters per line
\usepackage{framed}
\definecolor{shadecolor}{RGB}{248,248,248}
\newenvironment{Shaded}{\begin{snugshade}}{\end{snugshade}}
\newcommand{\AlertTok}[1]{\textcolor[rgb]{0.94,0.16,0.16}{#1}}
\newcommand{\AnnotationTok}[1]{\textcolor[rgb]{0.56,0.35,0.01}{\textbf{\textit{#1}}}}
\newcommand{\AttributeTok}[1]{\textcolor[rgb]{0.77,0.63,0.00}{#1}}
\newcommand{\BaseNTok}[1]{\textcolor[rgb]{0.00,0.00,0.81}{#1}}
\newcommand{\BuiltInTok}[1]{#1}
\newcommand{\CharTok}[1]{\textcolor[rgb]{0.31,0.60,0.02}{#1}}
\newcommand{\CommentTok}[1]{\textcolor[rgb]{0.56,0.35,0.01}{\textit{#1}}}
\newcommand{\CommentVarTok}[1]{\textcolor[rgb]{0.56,0.35,0.01}{\textbf{\textit{#1}}}}
\newcommand{\ConstantTok}[1]{\textcolor[rgb]{0.00,0.00,0.00}{#1}}
\newcommand{\ControlFlowTok}[1]{\textcolor[rgb]{0.13,0.29,0.53}{\textbf{#1}}}
\newcommand{\DataTypeTok}[1]{\textcolor[rgb]{0.13,0.29,0.53}{#1}}
\newcommand{\DecValTok}[1]{\textcolor[rgb]{0.00,0.00,0.81}{#1}}
\newcommand{\DocumentationTok}[1]{\textcolor[rgb]{0.56,0.35,0.01}{\textbf{\textit{#1}}}}
\newcommand{\ErrorTok}[1]{\textcolor[rgb]{0.64,0.00,0.00}{\textbf{#1}}}
\newcommand{\ExtensionTok}[1]{#1}
\newcommand{\FloatTok}[1]{\textcolor[rgb]{0.00,0.00,0.81}{#1}}
\newcommand{\FunctionTok}[1]{\textcolor[rgb]{0.00,0.00,0.00}{#1}}
\newcommand{\ImportTok}[1]{#1}
\newcommand{\InformationTok}[1]{\textcolor[rgb]{0.56,0.35,0.01}{\textbf{\textit{#1}}}}
\newcommand{\KeywordTok}[1]{\textcolor[rgb]{0.13,0.29,0.53}{\textbf{#1}}}
\newcommand{\NormalTok}[1]{#1}
\newcommand{\OperatorTok}[1]{\textcolor[rgb]{0.81,0.36,0.00}{\textbf{#1}}}
\newcommand{\OtherTok}[1]{\textcolor[rgb]{0.56,0.35,0.01}{#1}}
\newcommand{\PreprocessorTok}[1]{\textcolor[rgb]{0.56,0.35,0.01}{\textit{#1}}}
\newcommand{\RegionMarkerTok}[1]{#1}
\newcommand{\SpecialCharTok}[1]{\textcolor[rgb]{0.00,0.00,0.00}{#1}}
\newcommand{\SpecialStringTok}[1]{\textcolor[rgb]{0.31,0.60,0.02}{#1}}
\newcommand{\StringTok}[1]{\textcolor[rgb]{0.31,0.60,0.02}{#1}}
\newcommand{\VariableTok}[1]{\textcolor[rgb]{0.00,0.00,0.00}{#1}}
\newcommand{\VerbatimStringTok}[1]{\textcolor[rgb]{0.31,0.60,0.02}{#1}}
\newcommand{\WarningTok}[1]{\textcolor[rgb]{0.56,0.35,0.01}{\textbf{\textit{#1}}}}
\usepackage{graphicx}
\makeatletter
\def\maxwidth{\ifdim\Gin@nat@width>\linewidth\linewidth\else\Gin@nat@width\fi}
\def\maxheight{\ifdim\Gin@nat@height>\textheight\textheight\else\Gin@nat@height\fi}
\makeatother
% Scale images if necessary, so that they will not overflow the page
% margins by default, and it is still possible to overwrite the defaults
% using explicit options in \includegraphics[width, height, ...]{}
\setkeys{Gin}{width=\maxwidth,height=\maxheight,keepaspectratio}
% Set default figure placement to htbp
\makeatletter
\def\fps@figure{htbp}
\makeatother
\setlength{\emergencystretch}{3em} % prevent overfull lines
\providecommand{\tightlist}{%
  \setlength{\itemsep}{0pt}\setlength{\parskip}{0pt}}
\setcounter{secnumdepth}{-\maxdimen} % remove section numbering
\ifLuaTeX
  \usepackage{selnolig}  % disable illegal ligatures
\fi

\begin{document}
\maketitle

\hypertarget{preguntas-de-esta-tarea}{%
\section{Preguntas de esta Tarea}\label{preguntas-de-esta-tarea}}

\hypertarget{pregunta-1}{%
\subsubsection{Pregunta 1}\label{pregunta-1}}

Crea un vector llamado Harry formado por la sucesión de números
consecutivos entre el -10 y 27. Pídele a R que devuelva el elemento de
índice 7. Escribe el resultado.

\begin{Shaded}
\begin{Highlighting}[]
\NormalTok{Harry }\OtherTok{=} \FunctionTok{c}\NormalTok{(}\SpecialCharTok{{-}}\DecValTok{10}\SpecialCharTok{:}\DecValTok{27}\NormalTok{)}
\NormalTok{Harry[}\DecValTok{7}\NormalTok{]}
\end{Highlighting}
\end{Shaded}

\begin{verbatim}
[1] -4
\end{verbatim}

\hypertarget{pregunta-2}{%
\subsubsection{Pregunta 2}\label{pregunta-2}}

Da el máximo de la sucesión \(100\cdot 2^n -7\cdot3^n\) con
\(n=0,\dots,200\).

\begin{Shaded}
\begin{Highlighting}[]
\NormalTok{n }\OtherTok{=} \FunctionTok{seq}\NormalTok{(}\DecValTok{0}\NormalTok{, }\DecValTok{200}\NormalTok{, }\AttributeTok{by =} \DecValTok{1}\NormalTok{)}

\NormalTok{funcion }\OtherTok{=} \ControlFlowTok{function}\NormalTok{(x)\{}\DecValTok{100}\SpecialCharTok{*}\DecValTok{2}\SpecialCharTok{\^{}}\NormalTok{x }\SpecialCharTok{{-}} \DecValTok{7}\SpecialCharTok{*}\DecValTok{3}\SpecialCharTok{\^{}}\NormalTok{x\}}

\NormalTok{vec }\OtherTok{=} \FunctionTok{sapply}\NormalTok{(n, funcion)}
\FunctionTok{max}\NormalTok{(vec)}
\end{Highlighting}
\end{Shaded}

\begin{verbatim}
[1] 1499
\end{verbatim}

\hypertarget{pregunta-3}{%
\subsubsection{Pregunta 3}\label{pregunta-3}}

Crea la sucesión de números consecutivos entre 0 y 40. A continuación,
crea el vector 3 · 5n − 1 con n = 0, . . . , 40. Ponle como nombre x.
Ahora, da el subvector de los elementos que son estrictamente mayores a
3.5.

\begin{Shaded}
\begin{Highlighting}[]
\NormalTok{suc }\OtherTok{=} \FunctionTok{seq}\NormalTok{(}\DecValTok{0}\NormalTok{, }\DecValTok{40}\NormalTok{, }\AttributeTok{by=}\DecValTok{1}\NormalTok{)}
\NormalTok{x }\OtherTok{=} \FunctionTok{sapply}\NormalTok{ (suc, }\AttributeTok{FUN =} \ControlFlowTok{function}\NormalTok{(x)\{(}\DecValTok{3}\SpecialCharTok{*}\DecValTok{5}\SpecialCharTok{*}\NormalTok{x)}\SpecialCharTok{{-}}\DecValTok{1}\NormalTok{\}) }
\NormalTok{x[x }\SpecialCharTok{\textgreater{}} \FloatTok{3.5}\NormalTok{]}
\end{Highlighting}
\end{Shaded}

\begin{verbatim}
 [1]  14  29  44  59  74  89 104 119 134 149 164 179 194 209 224 239 254 269 284
[20] 299 314 329 344 359 374 389 404 419 434 449 464 479 494 509 524 539 554 569
[39] 584 599
\end{verbatim}

\hypertarget{pregunta-4}{%
\subsubsection{Pregunta 4}\label{pregunta-4}}

Crea una función que devuelva la parte real, la imaginaria, el módulo,
el argumento y el conjugado de un número, mostrando solo 2 cifras
significativas. RECOMENDACIÓN: En algún momento hará falta utilizar
vectores

\begin{Shaded}
\begin{Highlighting}[]
\NormalTok{fun }\OtherTok{=} \ControlFlowTok{function}\NormalTok{(z)\{}
\NormalTok{  real }\OtherTok{=} \FunctionTok{Re}\NormalTok{(z)}
\NormalTok{  img }\OtherTok{=} \FunctionTok{Im}\NormalTok{(z)}
\NormalTok{  mod }\OtherTok{=} \FunctionTok{Mod}\NormalTok{(z)}
\NormalTok{  ar }\OtherTok{=} \FunctionTok{Arg}\NormalTok{(z)}
\NormalTok{  con }\OtherTok{=} \FunctionTok{Conj}\NormalTok{(z)}
  
\NormalTok{  return }\OtherTok{=} \FunctionTok{c}\NormalTok{(real, img, mod, ar, con)}
\NormalTok{\}}

\NormalTok{z1 }\OtherTok{=} \FunctionTok{fun}\NormalTok{(}\DecValTok{5}\SpecialCharTok{+}\NormalTok{3i)}

\NormalTok{z1}
\end{Highlighting}
\end{Shaded}

\begin{verbatim}
[1] 5.0000000+0i 3.0000000+0i 5.8309519+0i 0.5404195+0i 5.0000000-3i
\end{verbatim}

\end{document}
